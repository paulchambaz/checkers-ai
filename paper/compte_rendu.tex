\documentclass[a4paper,11pt]{article}

\usepackage[margin=2cm]{geometry}
\usepackage{caption}
\usepackage{bookmark}
\usepackage{hyperref}
\usepackage{blindtext}
\usepackage{multicol}
\usepackage{titlesec}
\usepackage{amsfonts}
\usepackage{amsmath}
\usepackage{titling}

\title{\textbf{Application de l'Élagage Alpha-Bêta au Jeu de Dames}}
\author{Moncef Bouhabei, Paul Chambaz}
\date{Janvier 2023}

\titleformat{\section}{\large\bfseries}{\thesection. }{0em}{}
\titleformat{\subsection}{\normalsize\bfseries}{\thesubsection. }{0em}{}

\setlength{\headsep}{0cm}
\setlength{\columnsep}{1cm}

\begin{document}

\maketitle

Ce projet a été mené dans le cadre de l'Unité d'Enseignement
\emph{d'Intelligence Artificielle}, enseigné par \emph{Elise Bonzon}, dans le
cursus de licence 3 à \emph{l'Université Paris Cité}. L'objectif principal de ce
projet est de développer une intelligence artificielle capable de résoudre des
jeux à deux joueurs, en proposant une implémenatation de l'algorithme
alpha-bêta. Ce projet vise l'exploration de cet algorithme pour produire une
intelligence artificielle capable de joueur contre des joueurs de différents
niveaux. Le jeu des dames a été choisi en raison de sa complexité stratégique et
de la varitété des méthodes d'évaluation de la position. En effet, malgré des
règles simple, ce jeu permet l'élaboration de stratégies sophistiqués chez les
joueurs de haut niveaux. Nous espérons que notre programme sera capable de
reproduire certaines de ces stratégies, voire de les contrer. De plus, de
nombreuses possibilités existent pour évaluer les positions, cela permet de
développer une fonction d'évaluation complexe.

\hrulefill

\begin{multicols}{2}[]

\section{Description du jeu et du programme}

  \subsection{Motivation}

    Dans ce projet, nous avons choisi le jeu de dames. Ce choix a été motivé par
    plusieurs facteurs. Tout d'abord, le jeu de dames est relativement simple à
    implémenter, ce qui nous permet de nous concentrer sur les aspects les plus
    importants de notre algorithme plutôt que sur la complexité du jeu lui-même.
    Cependant, malgré une apparence simple, le jeu de dames comporte une
    profondeur stratégique considérable. En effet, même si elle ne peut être
    comparée à celle de jeux classiques comme les échecs ou le go, il offre un
    nombre coup potentiels de $5 \times 10^{20}$. Il a fallu attendre 1994 pour
    qu'un algorithme de jeu de dames remporte le titre de champion du monde de
    ce jeu, et 2007 pour qu'une résolution complète soit atteinte.

    Deuxièmement, et contrairement à d'autres projet envisagé, le jeu de dames
    nous apparu comme riche pour le nombre de méthode utilisée dans la fonction
    d'évaluation. En effet, en regardant une partie se dérouler, il à l'air
    facile de pouvoir juger de la force de la position de tel ou tel joueur.
    Bien sûr, si ce type de jugement est loin d'être suffisant pour
    l'analyse du jeu à très haut niveau, il reste cependant probable que la
    fonction d'évaluation nécessaire au fonctionnement de l'élagage alpha-bêta
    permette une solution plus créative et avec plus de résultat que d'autres
    jeux sur lesquel il est plus difficile de correctement juger d'une
    position.

    Finalement, en faisant des recherches pour ce sujet, nous avons appris
    l'importance de ce jeu dans l'histoire de l'intelligence artificielle.  Dès
    1952, un premier algorithme fut écrit. En 1959, le premier document
    utilisant le terme de \emph{Machine Learning} s'intitulais \emph{Some
    Studies in Machine Learning Using the Game of Checkers}. De plus avec le
    programme \emph{Chinook}, depuis 2007 le jeu de dames fait parti des jeu
    complètement résolu par l'intelligence artificielle. Ce projet, bien plus
    modeste ne s'interessera qu'à une application de l'élagage alpha-bêta.

  \subsection{Les règles du jeu de dames}

    Le jeu de dames se joue sur un damier de huit fois huit cases. Ce damier
    comporte deux type de cases, des cases blanches et des cases noires, les
    pions durant la partie ne pourront se déplacer que sur les cases noires. Ce
    jeu se joue avec deux joueurs, l'un prendra les pions blancs et les pions
    noirs.

    Au début de la partie, chaque joueur reçoit exactement douze pions qu'il
    doit placer sur les cases noires des trois premières lignes. À la fin du
    placement initial, seule les deux lignes du milieu ne devrait pas avoir de
    pion. Le joueur blanc peut alors commencer.

    Un pion au dame n'a le droit d'avancer qu'en diagonale de une seule case
    vers l'avant. Un pion a donc deux cases potentiels sur lesquels il peut
    aller.  Cependant, si un pion allié ou ennemi se situe sur une des cases
    sur lequel se trouve sur la case ou le pion veut se déplacer alors ce
    déplacement est interdit. Une fois que le pion a été déplacé, le tour passe
    à l'autre joueur qui peut effectuer à son tour un coup.

    Un pion peut manger un autre pion. Pour ce faire, si un pion ennemi se
    situe en diagonale du pion, et que la case derrière en diagonale n'est pas
    occupée, alors le pion peut le manger. Si c'est le cas, alors le pion se
    déplacera sur la case libre de derrière et le pion ennemi sera retiré du
    damier -- le jouer ne pourra plus l'utiliser. Il faut noter que du fait, un
    pion ne peut manger qu'en avant par rapport à son point de départ. De plus,
    si un pion peut manger alors un pion doit manger, si plusieurs choix sont
    disponible alors, le choix est laissé au joueur. Finalement, si un joueur a
    mangé un pion ennemi, alors il conserve son tour et peut effectuer un autre
    coup si et seulement si le même pion peut continuer à manger.

    Un pion peut devenir une reine, d'où le jeu tire son nom. Pour ce faire, il
    faut alors qu'il arrive à l'autre bout du plateau. Une fois cette
    destination atteinte, le pion est designé comme reine (souvent en placant
    un autre pion par dessus le premier). Une reine contrairement à un pion
    peut se déplacer dans les deux sens et sans restriction de case. Une dame
    peut donc aller d'un côté à un autre du damier. Une dame peut toujours
    manger mais elle ne peut le faire que si le pion ennemi est immédiatement à
    côté d'elle. Autrement dit, une dame ne peut pas parcourir deux cases, puis
    manger. En revanche, une dame n'a pas de limite sur la quantité de pièce.

    Le jeu se termine dans deux cas. Premièrement, si un des joueurs n'a plus
    de pièces, alors il perd la partie. Deuxièmement, si un des joueurs n'a
    plus de coups valable, alors il perd aussi la partie.

  \subsection{Description du programme}

    Pour ce projet, nous avons fait le choix d'utiliser le langage \emph{lisp}.
    Ce langage ayant été initialement créé pour résoudre des problèmes
    d'intelligence artificielle, nous nous sommes dit que ce choix serait
    adapté. De plus, ce langage, souvent décrit comme \emph{méta-programmable},
    permet de modifier la syntaxe du programme durant l'execution du programme.
    Cela permet de modéliser des cas mathématiques abstraits ou pour notre
    problème, des états et actions abstraits. Dans notre cas, nous utilisons le
    \emph{common-lisp}, avec la librairie \emph{cl-sdl2} pour l'implémentation
    graphique du logiciel. Les algorithmes d'intelligence artificielle ont été
    implémenté. Finalement, nous avons aussi choisi \emph{lisp} pour une
    raison, même si ce programme est un programme qui est souvent interprété,
    et qu'il possède un garbage-collector, il est généralement considéré comme
    rapide. Nous espèrons que ce choix permettra de pouvoir explorer plus de
    possibilité dans l'arbre de recherche.

    Le programme se découpe en plusieurs fichiers:

  \begin{itemize}
    \item \emph{packages.lisp} contient les informations relative au paquet
      contenant l'application.
    \item \emph{main.lisp} contient la mise en place du \emph{main loop} de
      l'application. Il met en place \emph{sdl2}, mais ne contient pas
      d'implémentation complexe qui sont laissé aux différents paquets.
    \item \emph{constants.lisp} contient une liste des constantes requises pour
      le programme. La plupart des constantes sont utilisées pour l'interface
      graphique. D'autres sont lié à la logique du jeu et d'autres encore sont
      liées à l'intelligence artificielle.
    \item \emph{interface.lisp} contient toutes les fonctions requises pour
      l'interface graphique. Dans ce fichier est géré à la fois l'affichage et
      l'interaction avec le joueur.
    \item \emph{logic.lisp} contient tout le code permettant le bon déroulement
      du jeu. C'est dans ce paquet qu'est défini formellement des concepts comme
      l'état du jeu, une action ou l'ensemble des actions permisent à partir
      d'un état.
    \item \emph{intelligence.lisp} contient tout le code de l'implémentation
      alpha-bêta et les fonctions nécessaire pour le bon fonctionnement de
      l'intelligence artificielle.
      % ajouter training.lisp
  \end{itemize}

\section{Description de l'intelligence articielle}

  \subsection{Chinook}

    Avant de commencer la description du programme de ce projet, nous nous
    interesseront au programme Chinook. Ce programme est connu pour avoir battu
    le record du monde en 1994.\footnote{Chinook devint champion sans avoir
    battu \emph{Marion Tinsley} qui était considéré comme de loin le meilleur
    joueur du monde, détenant le titre pendant plus de 40 ans. Le joueur avait
    pris sa retraite suite à des problèmes de santé et décedera l'année
    suivante invaincu !} En 2007, cependant, les chercheurs de
    \emph{l'Université d'Alberta} réussirent à résoudre le problèmes du jeu de
    dames. Si le programme que nous proposons est loin d'égaler cette
    performance, il est important d'étudier le fonctionnement de
    \emph{Chinook}.  En effet, ce programme n'est pas un programme du type
    \emph{réseau de neuronnes profonds}, comme le seront plus tard des
    programmes révolutionnaires comme \emph{Deep Blue} qui vaincu \emph{Garry
    Kasparov}, le champion du monde des échecs. \emph{Chinook} est bien plus
    propre du fonctionnement de l'algorithme utilisé dans ce projet.  Le
    programme fonctionne en trois phases distinctes, la première phase est
    résolu avec un base de donnée contenant tous les coups classiques utilisés
    par les joueurs professionels. La seconde phase qui nous intéresse
    particulièrement, est résolu par un algorithme de recherche profonde. Ce
    terme désigne les algorithmes basé sur le \emph{depth-first search} qui
    explore des graphes. Ce type d'algorithme est très similaire à l'algorithme
    alpha-bêta utilisé dans ce projet. De plus l'algorithme utilisé par
    \emph{Chinook} utilise une fonction d'évaluation pour juger d'une certaine
    position. Finalement, sur plusieurs années, les chercheurs ont compilé une
    base de coups pour effectuer des algorithmes pour la dernière partie du jeu
    en calculant pendant plusieurs années, pour obtenir la liste des actions
    parfaites pour les huit dernier coups. On notera qu'il s'agit probablement
    d'un algorithme \emph{minimax} que l'on laisse tourner sur toutes les
    configurations légale de fin du jeu.

  \subsection{PEAS du programme}

  % donner la definition formelle du probleme de recherche
  % etat, action, etc...

  \subsection{Définition formelle d'un état}

    Il nous faut tout d'abord définir formellement l'état. Tout d'abord, il
    faut noter que la définition d'un état doit contenir toute les informations
    nécessaire pour pouvoir calculer indépendamment les décisions sur le futurs
    du jeu. Notons aussi que c'est cet état qui sera passé à l'arbre. Le plus
    compacte nous somme donc capable de représenter cet état, le plus petit le
    coup mémoire associé sera -- pas au sens de la compléxité, mais simplement
    au sens de la vitesse d'éxecution en cycle de processeur. De la notion de
    \emph{cache-locality}, on espère que cela aura aussi un impact sur la
    vitesse d'execution du programme. Finalement, pour mieux utiliser la notion
    de récursivité, il serait intéressant de ne jamais vraiment stocker l'état,
    mais simplement de le transformer en le faisant passer. On n'aurait alors
    jamais deux copies : cela permettra de réduire considérablement l'empreinte
    mémoire du programme. Il faut donc que l'état soit facilement réversible.
    Il faut qu'une la fonction d'action puisse être une \emph{bijection} entre
    un état et un autre.

    Nous avons donc choisi de représenter l'état comme une liste de troix
    élements:

    \begin{itemize} 
      \item Le tableau du damier, comportant un total de $8 \times 8$ cases. A
        chaque case, un entier est attribué, $0$ pour une case vide, $1$ pour
        un pion noir, $2$ pour un pion blanc, $3$, pour une dame noire et $4$
        pour une dame blanche. On fait le choix de stocker les cases blances :
        en effet, on espère que cela permette à certains algorithme d'être plus
        rapide et la taille total de la représentation de la liste ne sera pas
        copié, ce qui devrait ne pas couter plus.  
      \item Le numero du joueur, $0$ pour les noirs et $1$ pour les blancs.
      \item Un argument optionel si un pion doit continuer à manger. Ce genre
        de configuration n'arrive que après qu'un pion ait mangé et qu'il peut
        encore manger après. Comme le pion peut faire un choix, nous avons
        choisi de laisser le programme chercher et considerer cela comme un
        autre tour.  Durant ce tour, le rôle du joueur ne changera pas, mais il
        faut indiquer à l'état que seul ce pion peut jouer pour obtenir la
        liste correcte des actions légales. Si on est dans cette situation, un
        entier désignant le numéro de la pièce par ordre d'apparition sur le
        damier du joueur.
    \end{itemize}

    Si on utilise un octet pour chacun de ces trois paramètres constituant
    l'état, alors stocker l'état revient à stocker 66 octets.

    % TODO cette section peut probablement être réecrite avec un damier de 32
    % cases seulement pour reduire le profil mémoire.

    D'un point de vue plus formel, on note l'état: 

    \[ E_{n} \in \mathbb{E} \]

    Ou $\mathbb{E}$ est l'ensemble de tous les états possibles de position
    légales sur le damier. On ajoute aussi $E_{n}$ afin de pouvoir donner un
    sens à la direction du jeu.

  \subsection{Définition formelle d'une action}

    Une action dans ce programme doit représenter le déplacement d'un pion à un
    autre pion, il doit aussi donner des informations sur le joueur qui vient de
    jouer afin d'être completement bijectif. Une action est donc:

    \begin{itemize}
      \item Un premier entier signifiant la position de départ du pion.
      \item Un second entier signifiant la position d'arrivée du pion.
      \item Un entier qui signifie le joueur qui a fait l'action.
    \end{itemize}

    D'un point de vue plus formel, on note une action:

    \[  a \in \mathbb{A} \]
    \[ u : \mathbb{A} \times \mathbb{E} \to \mathbb{A} \times \mathbb{E}, 
        a, E_{n} \mapsto a, E_{n+1} \]
    \[ u^{-1}(u(a, E_{n})) = a, E_{n} \]
    \[ \mathcal{A} \subset \mathbb{A}, m \in \mathbb{N} \]
    \[ v : \mathbb{E} \to \mathcal{A},
        E_{n} \mapsto \left\{ a_{1}, ..., a_{m} \right\} \]

    Ici $\mathbb{A}$, est l'ensemble des actions légales et $u$ est la fonction
    qui transforme un état en un autre à partir d'une action en allant vers
    l'avant. On renvoie aussi l'action afin de pouvoir avoir une fonction
    bijective.

    Finalement, $v$ est la fonction qui génère tous les coups légaux. Elle
    renvoie une liste d'action légale. $\mathcal{A}$ est un sous ensemble de 
    $\mathbb{A}$ et m est le cardinal de $\mathcal{A}$.
    
    Nous utiliserons toutes ces fonctions pour traverser le graphe de l'ensemble
    des actions et générer l'arbre de recherche. On utilise $v$ pour générer
    l'ensemble des actions légale, puis pour chaque membre de la liste, on
    avance dans le graphe avec $u$. Pour remonter, il suffit d'utiliser
    $u^{-1}$.

    Du point de vue de l'implémentation, il est très important que ces fonctions
    aient un empreinte mémoire \emph{locale}, c'est à dire que l'on ne veut
    jamais stocker leur résultat, mais toujours le faire passer au reste du
    programe. Cela empechera de consommer de l'espace mémoire au cours du
    graphe. De plus, afin d'accélerer la vitesse d'éxecution du programme, il
    est important que ces fonctions soit rapide puisqu'elle seront effectuée
    pour chaque noeud de l'arbre au moment du déroulement.

  \subsection{La fonction d'évaluation}

    La fonction d'évaluation est très important dans ce programme:
    
    \[ f : \mathbb{E} \to \mathbb{R},
      E_{n} \mapsto f(E_{n}) \]
    \[ f(E_{n}) = w_{1} \times f_{1}(E_{n}) + ... + w_{p} \times f_{p}(E_{n}) \]

    On voit bien que la fonction $f$ est en fait composé de $p$ fonctions
    $f_{p}$. Chacune de ces fonctions est en fait une fonction d'évaluation elle
    même qui est définie comme $f$. $f$ est donc la composée de $p$ fonctions
    d'évaluation elle même.

    De façon plus concrête, chaque fonction $f_{p}$ est une \emph{évaluation de
    l'état}. Cela ne veut rien dire de plus, il n'y a donc, en particulier pas
    de notion de \emph{bon état} et de \emph{mauvais état}. Alors, comment
    fait-on pour discriminer dans $f$ pour obtenir la notion désiré de bon état.
    C'est l'utilité des $w_{p}$.

    Les $w_{p}$ sont une liste de poids que l'on utilise pour attribuer à chaque
    fonction $f_{p}$ un poids. Ainsi, si le poids est positif et grand, alors la
    fonction d'évaluation $f_{p}$ est un particulièment bon moyen de juger de la
    bonne qualité de la position sur le damier. À l'inverse, si le poids est
    négatif et grand, alors la fonction d'évaluation $f_{p}$ est un
    particulièrement bon moyen de juger de la mauvaise qualité de la position
    sur le damier. En revanche si le poids est proche de $0$, alors la mesure
    n'est pas vraiment un bon moyen de juger de la qualité d'une position sur le
    damier.

    Pour ce projet, nous avons fait le pari que nous ne savions pas vraiment ce
    qui était bon pour chaque fonction d'évaluation, dans la prochaine section,
    nous verrons comment nous avons pu trouver ces poids.

    Pour la liste des potentiels fonctions d'évaluations $f_{p}$, nous avons
    chercher une liste de fonctions qui pourraient être une bonne mesure. En
    effet, si le poids venait à être autour de 0, alors c'est possible que cette
    fonction soit plus un ralenti qu'une aide. En effet, cette fonction est
    executé pour chaque noeud de l'arbre, en la rendant plus rapide, cela permet
    d'explorer plus de noeud. Nous avions donc l'objectif suivant, dans un
    premier temps énumerer une liste de fonctions d'évaluations qui nous
    paraissaient prometteuse. Dans un second temps, nous ajustions les poids.
    Dans un troisième temps, nous retirions les fonctions d'évaluations qui
    n'avaient pas fait leur preuve et ne gardions que les fonctions qui étaient
    suiffisement efficace.

    Au départ, nous avons identifié fonctions d'évaluations:

    \begin{enumerate}
      \item Le nombre de pièces du joueur
      \item Le nombre de pièces du joueur adverse
      \item Le nombre de reine du joueur
      \item Le nombre de reine du joueur adverse
    \end{enumerate}

  \subsection{Un tournoi pour ``l'entraînement'' de la fonction d'évaluation}

    Pour trouver la valeur exacte de chaque poids, nous somme parti du principe
    que nous ne pourrions pas trouver en raisonnant. Bien sûr, nous pourrions
    mettre des poids arbitraires, mais comment s'assurer que nous avions raison
    et qu'une évaluation que nous pensions bonne était en fait pas vraiment
    utile, voir pire, mauvaise. Du fait nous avons eu l'idée de faire une sorte
    de tournoi pour juger de la qualité des poids.

    Au départ, chaque poids est mis à zero, une suite de dix tirages entre -1 et
    1 sont ensuite réalisé et additioné pour produire le poids initial de chaque
    fonction d'évaluation. Nous générons ensuite un nombre $n$ de fonctions
    d'évaluations différentes $f$. Nous réalisons ensuite un genre de tournoi de
    tennis, ou chaque ia en affronte une autre, puis le vainqueur d'une autre
    affrontement jusqu'a ce que la meilleure soit trouvé. Chaque affrontement
    comporte 5 parties, si il y a match nul à la fin de ces 5 parties, alors une
    des ias est choisies au hasard.

    Cela produit ensuite une base sur laquelle on peut ensuite créer notre
    prochain tournoi. Sur la base des poids du gagnant, on peut refaire les dix
    tirages entre -1 et 1 pour recréer $n$ ias et refaire un second tournoi. On
    mesure ensuite la distance entre les poids de la génération 1 et les poids
    de la génération 2, si ces poids sont inferieurs à une certaine valeur,
    alors on arrête l'entrainement, en revanche, si ce n'est pas encore le cas,
    on continue à regenerer des nouvelles ia. Plus précisement, nous avons
    arrêter lorsque l'on observait une stabilisation de la différence.

    Ce type de méthode s'apparent vaguement à un algorithme génétique, mais on
    prend toujours uniquement le meilleur de chaque génération et on ne croise
    pas les résultats, du fait il n'en est pas vraiment un.

  \subsection{Notion de difficulté}

\section{Bilan du projet}

\end{multicols}

\end{document}
